\documentclass{beamer}

\usepackage{fontspec,xunicode,xltxtra}

\usepackage{tikz}

\XeTeXlinebreaklocale "zh"
\XeTeXlinebreakskip = 0pt plus 1pt minus 0.1pt

%\setmainfont[Mapping=tex-text]{AR PL UMing CN:style=Light}
%\setmainfont[Mapping=tex-text]{AR PL UKai CN:style=Book}
%\setmainfont[Mapping=tex-text]{WenQuanYi Zen Hei:style=Regular}
%\setmainfont[Mapping=tex-text]{WenQuanYi Zen Hei Sharp:style=Regular}
%\setmainfont[Mapping=tex-text]{AR PL KaitiM GB:style=Regular} 
%\setmainfont[Mapping=tex-text]{AR PL SungtiL GB:style=Regular} 
%\setmainfont[Mapping=tex-text]{WenQuanYi Zen Hei Mono:style=Regular} 

\newfontfamily\hei{WenQuanYi Micro Hei}
\newfontfamily\whei{WenQuanYi Zen Hei}
\newfontfamily\kai{AR PL UKai CN}
\newfontfamily\song{AR PL UMing CN}
\newfontfamily\bhei{cwTeXHeiBold}
%\newfontfamily\lishu{SIMLI}
\setmainfont[Mapping=tex-text]{WenQuanYi Micro Hei}
\setsansfont[Mapping=tex-text]{AR PL UKai CN}
\setmonofont[Mapping=tex-text]{WenQuanYi Zen Hei Mono}

\renewcommand{\baselinestretch}{1.25}


\mode<presentation>
{
  \usetheme{Warsaw}
  % or ...
  %\usetheme{default}

  \setbeamercovered{transparent}
  % or whatever (possibly just delete it)
}


\usepackage[english]{babel}
% or whatever

%\usepackage[latin1]{inputenc}
% or whatever

\title[MS2023] % (optional, use only with long paper titles)
{\Huge 数学软件}

\subtitle
{第一讲:准备工作} % (optional)

\author[Wang HY] % (optional, use only with lots of authors)
{王何宇}
% - Use the \inst{?} command only if the authors have different
%   affiliation.

\institute[Zhejiang University] % (optional, but mostly needed)
{
  浙江大学数学科学学院\\
  计算数学系
}
% - Use the \inst command only if there are several affiliations.
% - Keep it simple, no one is interested in your street address.

\date[Short Occasion] % (optional)
{}


% If you have a file called "university-logo-filename.xxx", where xxx
% is a graphic format that can be processed by latex or pdflatex,
% resp., then you can add a logo as follows:

\pgfdeclareimage[height=1cm]{university-logo}{images/zju.jpg}
\logo{\pgfuseimage{university-logo}}

% Delete this, if you do not want the table of contents to pop up at
% the beginning of each subsection:
%\AtBeginSubsection[]
%{
%  \begin{frame}<beamer>{Outline}
%    \tableofcontents[currentsection,currentsubsection]
%  \end{frame}
%}


% If you wish to uncover everything in a step-wise fashion, uncomment
% the following command: 

%\beamerdefaultoverlayspecification{<+->}


\begin{document}

\begin{frame}
 \titlepage
\end{frame}
%\begin{frame}{Outline}
%  \tableofcontents
  % You might wish to add the option [pausesections]
%\end{frame}

\begin{frame}{自我介绍}
  \begin{itemize}
  \item 王何宇, 浙江大学数学科学学院, 信息与计算科学系.
  \item email: wangheyu@zju.edu.cn
  \item 手机: 13456940632
  \item 我承担很多信息与计算方向的专业课程, 所以大概率会和大家共事 3 年甚至更多.
  \end{itemize}
\end{frame}

\begin{frame}{使用计算机的基本原则}
  \begin{itemize}
  \item<1-> 用计算机提升你的工作和学习效率,而不是相反;
  \item<2-> 白嫖使人快乐! 
  \item<3-> 参与和回馈社区;
  \item<4-> 如果你是计算数学,应用数学和统计的同学,像重视黑板那样重视计算机;
  \item<5-> 一切都在非线性变化,保持清醒和独立思考,只学对你现实有用的计算机技能。
  \end{itemize}
\end{frame}

\begin{frame}{学习本课的基本条件}
  \begin{itemize}
  \item<1-> 一个可以正常运行的 Linux 系统, 可以是独立系统, 双系统, 云主机, 或虚拟机.
  \item<2-> 能正确使用键盘, 打字速度不宜太低.
  \item<3-> 未来计划走应用数学的发展方向.
  \item<4-> 如果以上条件不能满足, 也不准备克服, 建议立刻退课或换课, 以免浪费时间.
  \end{itemize}
\end{frame}

\begin{frame}{和计算机交流}
  \begin{columns}[c]
% create the column with the first image, that occupies
% half of the slide
    \begin{column}{.5\textwidth}
      \begin{itemize}
      \item<1-> 机器内部使用的是二进制数据流;
      \item<2-> 使用 shell: 打开终端,输入命令,得到反馈;
      \item<3-> 阅读和实现: Beginning Linux Programming 4th edtion, chapter 2.
      \item<4-> 通过脚本实现自动化操作.
      \end{itemize}
    \end{column}
% create the column with the second image, that also
% occupies half of the slide
    \begin{column}{.5\textwidth}
    \begin{figure}
        \centering
        \includegraphics<1>[width=0.75\textwidth]{../images/matrix.bmp}
        \includegraphics<1>[width=0.75\textwidth]{../images/blank.png}       
        \includegraphics<2->[width=0.75\textwidth]{../images/matrixbin.png}
        \includegraphics<2->[width=0.75\textwidth]{../images/shell.png}
    \end{figure}
    \end{column}
\end{columns}
  %% \begin{tikzpicture}
  %%   \node (img1) {\includegraphics[height=3cm]{res/matrix.bmp}};
  %% \end{tikzpicture}
\end{frame}

\begin{frame}{文本编辑器}
  \begin{itemize}
  \item<1-> 能基本上完全用键盘控制.
  \item<2-> 适合终端和 shell 的模式, 不占用太多的软硬件和通讯资源.
  \item<3-> 能快速高效地发送和接受指令, 完成编码.
  \item<4-> 推荐: emacs, vim, vscode.
  \item<5-> 选择一款最适合你自己, 能把你的工作效率提到最高的编辑器.
  \item<6-> 如果你不知道怎么配置你的编辑器, 用缺省配置, 或者像同行要一个配置文件. 
  \end{itemize}
\end{frame}

\begin{frame}{保存你的数据}
  \begin{itemize}
  \item<1-> 用正确的方式存放你的重要数据:
    \begin{itemize}
    \item<1-> 用云盘存放你的学习资料;
    \item<1-> 用git仓库存放你的代码;
    \end{itemize}
  \item<2-> 程序员社区:github;
    \begin{itemize}
    \item<2-> 悲剧:被微软收购了;
    \item<2-> 悲剧:不能稳定访问;
    \item<2-> 替代品:gitee.
    \end{itemize}
  \item<3-> git, 一种代码管理方式;
  \item<4-> 选择一个可靠的商业云盘,推荐坚果云。
  \end{itemize}
\end{frame}

\begin{frame}{实际工作的例子:一个自己开发的小项目}
  \begin{itemize}
  \item<1-> 项目构成:程序源码,测试数据,测试结果,报告,幻灯片,参考文献...
  \item<2-> 项目生成:程序编译,运行,报告编译
  \item<3-> 项目管理:当某些源码或中间数据发生改变,需要及时更新
  \item<4-> 项目发布:提交到git网站上。
  \end{itemize}
\end{frame}

\begin{frame}{实际工作的例子:科学计算}
  \begin{itemize}
  \item<1-> 计算流体力学中的hello world: 方腔流
  \item<2-> 寻找合适的文献和模拟平台
  \item<3-> 构建代码,运行,测试
  \item<4-> 用合适的工具分析,观察数据,形成报告
  \item<5-> 现代计算机功能强大,即便是如此落后的虚拟机,也可以进行物理级别的模拟
  \end{itemize}
\end{frame}

\begin{frame}{作业}
  自行查阅适当的资料,完成下述工作:
  \begin{itemize}
  \item 创建你的gitee帐号,创建一个叫math-soft的源,将其设成公开仓库
  \item 在你的源内,增加一篇文档,名为:enviroment.tex,内容为对你参加课程学习的计算环境的描述,要求至少包含:
    \begin{itemize}
    \item 你的计算机型号,CPU型号,内存大小,硬盘大小,显卡型号 (指你真正的计算机)
    \item 你的Linux实现方式:主系统,双系统还是虚拟机,或者其他。如果是虚拟机,描述一下虚拟机的内存大小和硬盘大小。
    \item 你的Linux版本,和安装了哪些额外的软件(系统自带的不用列,自己安装的不论是否是我视频中有的尽量列出)。
    \item 你的编辑器和你的gcc编译器的版本。
    \item 用最多不超过200字评估一下你在未来的学习和工作中使用Linux环境工作的可能性和可能场景。
    \item 文章应该能用latex编译通过,可以参考我的文件,注意字体需要安装。最后在学在浙大提交你的git源地址(gitee那个)。
    \end{itemize}
  \end{itemize}
\end{frame}


\end{document}


