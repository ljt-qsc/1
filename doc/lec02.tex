%!Tex Program = xelatex
%\documentclass[a4paper]{article}
\documentclass[a4paper]{ctexart}
\usepackage{xltxtra}
\usepackage{url}
%\setmainfont[Mapping=tex-text]{AR PL UMing CN:style=Light}
%\setmainfont[Mapping=tex-text]{AR PL UKai CN:style=Book}
%\setmainfont[Mapping=tex-text]{WenQuanYi Zen Hei:style=Regular}
%\setmainfont[Mapping=tex-text]{WenQuanYi Zen Hei Sharp:style=Regular}
%\setmainfont[Mapping=tex-text]{AR PL KaitiM GB:style=Regular} 
%\setmainfont[Mapping=tex-text]{AR PL SungtiL GB:style=Regular} 
%\setmainfont[Mapping=tex-text]{WenQuanYi Zen Hei Mono:style=Regular} 


\title{第二讲: 进一步深入 Linux 社区}
\author{王何宇}
\date{}
\begin{document}
\maketitle
\pagestyle{empty}

\section{Linux 环境}

Linux 环境与其说是一些软件的使用, 更重要的是一些工作的原则和习惯. 这才是我们真正需要掌握的.
上一讲中, 我们已经体验了:

\begin{itemize}
\item 尽量使用键盘, 充分利用快捷键, 在字符界面下高效率地完成工作;
\item 用 \LaTeX 来组织你的科技文献, 使它看上去更加专业.  
\end{itemize}

不少同学因为初次接触, 因此会遇到很多实际的困难, 并且耗费了相当大的精力和时间.
但如果你要达到至少我这样一位老人家的工作效率, 这些付出是值得的. 但是千万不要忘记我们的原则:

\begin{itemize}
\item 一切都是经验和规范, 无关你的天赋和智商, 要多看文档, 多问, 善用搜索引擎;
\item 人不要去做机器该做的事情, 不要太勤劳.
\end{itemize}

今天, 我们进一步深入这个过程. 我们先来阅读一本靠谱的书籍的部分章节:

N. Matthew and R. Stones, Beginning Linux Programming,
4th Edition, Wiley Publishing, Inc.,
Indianapolis, 2008.

\subsection{Linux 的基本结构}

Linux 同样具有一个文件树系统, 也就是它的文件也全部在 \verb|/| 下展开. 我们可以用 \verb|cd| 命令进入这个目录,
\begin{verbatim}
cd /
ls
\end{verbatim}

这里 \verb|ls| 是列出当前目录下全部文件的意思. 可以看到, 基本上有这样一些目录:

\begin{itemize}
\item \verb|bin|, 系统级可执行文件, 比如 \verb|ls| 这个命令就在这里;
\item \verb|usr|, 用户级系统文件, 比如用户软件. 该目录下有二级目录, 比如 \verb|/usr/bin|,
  很多应用软件的启动程序就放在这里.
\item \verb|etc|, 系统的软硬件配置文件很多都在这里.
\item \verb|home|, 用户目录, 每一个二级目录都是一个用户的全部数据. 
\item ...
\end{itemize}

理论上说, 一个用户应该只修改和维护他个人目录中的文件, {\bf 除非知道你正在干什么,
  否则不要去修改除用户目录之外的任何文件!} 即便你知道你在干什么, 切记做好适当的备份或回滚策略.
此外, 这些目录功能的分配, 只是一种约定, 有很多例外. 从现在起, 你应该尽量将数据只存放在你的个人用户目录下,
它的全称是 \verb|\home\你的用户名|. 它有一个缩写: \verb|~|. 所以, \verb|cd ~|
就可以直接进入你的个人目录. 从现在起, 要谨慎使用 \verb|sudo| 命令, 它使你作为为管理员来发布命令,
管理员可以做任何事情, 比如删除根目录: \verb|rm / -fr| ({\bf 千万不要去尝试!})

你既是你自己机器的管理员, 也是你自己机器的用户. 请时刻对你当前的角色保持清醒. 当你只想做用户时,
不要做管理员该做的事;当你想做管理员时, 确保你自己知道自己在干啥! 举个例子: 安装和配置应用软件,
并让它成为系统功能的一部分, 是典型的管理员才做的事情. 你应该一次装好, 以后尽量少折腾. 我们不培养管理员.
另外, 请千万记住, 在 Linux 下, 不要动不动就重装整个系统. 一般再严重的错误,
都可以通过部分软件的调整和安装来修复. 不要轻易用关机的方式重启系统, 这有可能导致 Linux 系统崩溃.
实际上, 如果是安装在服务器或者工作站上的 Linux 系统, 经常会很久不关机, 不重启. 甚至有计算机,
一生都没关过机. 

Linux 的命令, 实际上是一个个可执行文件. 这些可执行文件分两类, 一类是二进制可执行文件,
另一类是批处理命令, 又称 shell 脚本文件. 它是一系列 Linux 命令结合在一起完成一个复杂的功能. 
Linux 的命令之前都必须跟一个称为路径的参数. 比如真正的 \verb|ls| 的完整命令是
\verb|/bin/ls|. 但是因为系统在环境变量 \verb|PATH| 中规定了 \verb|/bin| 是一个默认的路径,
所以 \verb|/bin| 下的所有命令, 都不需要输路径前缀了. 你可以用 \verb|env | grep PATH|
来观察你当前的 \verb|PATH| 变量的内容. 比如我现在的输出是:

\begin{verbatim}
PATH=/home/hywang/anaconda3/bin:/home/hywang/anaconda3/condabin
:/usr/local/sbin:/usr/local/bin:/usr/sbin:/usr/bin:/sbin:/bin
:/usr/games:/usr/local/games:/snap/bin
\end{verbatim}

这些目录之下的可执行文件, 都不需要路径前缀. 但是注意, 当前目录 \verb|.| 并不在路径中. 因此,
即便你就在命令所在的目录下, 但是这个目录不在路径中, 你也必须打路径前缀. 这就是我们学习 C 语言时,
得到的可执行文件, 运行必须加 \verb|./运行文件| 的原因.


退回上一层目录的命令是 \verb|cd ..|, 这里 \verb|..| 是上一层目录的意思.
你可以一直退到 \verb|/|.

Linux 的二进制可执行文件是怎么来的? 它其实就是 C 语言编译出来的. 当你构建文件 \verb|hello.c|:

\begin{verbatim}
#include <stdio.h>
#include <stdlib.h>
int main()
{
    printf(“Hello World\n”);
    exit(0);
}
\end{verbatim}

然后运行:

\begin{verbatim}
gcc -o hello hello.c
\end{verbatim}

以后, 你就得到一个可执行文件 \verb|hello|, 它本质上和一个 Linux 系统命令没有任何区别. 
你如果够无聊, 可以将它复制到缺省路径 \verb|/usr/local/bin| 下,  

\begin{verbatim}
sudo cp hello /usr/local/bin
\end{verbatim}

然后你就可以在任何目录下运行一个新命令 \verb|hello|. 

最后, 介绍两个很有用的命令: \verb|man| 和 \verb|info|. 它们提供 Linux 命令的官方文档. 比如:
\verb|man ls|. 这里 \verb|man| 是 manual 的缩写而不是男人.

\section{用 shell 脚本来加速}

首先介绍一个 UNIX 哲学: 一次做好一件具体的事情, 然后将零星的功能链接起来, 成为一个更加丰富的命令.
或者说: 作为系统架构师, 你在创建具体命令时, 要考虑到更高层次上的重用;而作为用户,
尽可能利用现有的命令的链接去完成更复杂的功能. 用互联网黑话说, 就是不要重复造轮子.

\subsection{管道和重定向}
比如: \verb|ls -l| 是列出当前目录下的全部文件, 但这样会导致一屏幕可能放不下.
\verb|more| 命令则是将一个超出一屏幕的输出, 在输出一满屏内容后暂停, 然后可以逐行(按回车)输出.
将两个命令用 \verb!|! 管道(pipe) 链接在一起: \verb!ls -l | more! 就是将当前目录下所有文件列出,
按屏停顿.

我们之前还有一个例子, \verb|env| 实际上是输出全部环境变量, 而 \verb|grep| 是过滤的意思,
所以 \verb!env | grep PATH! 就是输出全部环境变量中包含 \verb|PATH| 内容的东西.

这些组合都可以灵活搭配使用. UNIX 哲学在所有分支中都是通行的. 

大家可能注意到, shell 本身也是一个命令. 事实上, 目前流行的 shell 有多个版本, 我们使用的是 bash,
这也是开源社区最通行的版本. 使用 Mac OS 的同学可能注意到 Mac 的默认 shell 不是 bash.
这个命令本身的位置在 \verb|/bin/bash|. Linux 的命令实际上在 shell 中运行和反馈, shell 则提供 pipe
和环境变量等等这样的运行环境. 除了 pipe, shell 还有一个重要特性是重定向(redirection).

我们知道, C 语言中输出, 用 \verb|printf| 函数, 本质上是向一个被称为 \verb|stdout|
的设备文件输出字符流. \verb|printf| 和 \verb|stdout| 都在头文件 \verb|stdio.h| 
定义. 在正常情况下, \verb|stdout| 是定义在显示器上的. 但我们可以用重定向, 将其重定向为一个文本文件.
比如:
\begin{verbatim}
ls -l > re.txt
\end{verbatim}
将当前目录的内容重定向到 \verb|re.txt| 下. 这里 \verb|>| 每一次都会重新生成一个新的输出结果文件.
而 \verb|>>| 则是在文件结尾新增.
\begin{verbatim}
ls -l >> re.txt
\end{verbatim}

有的时候, 我们写了一个程序, 会有大量的输出, 如果从屏幕走, 记录本身就很麻烦.
这时你没有必要专门做一套文本文件的读写机制, 而只需要重定向就行了. 除了 stdout 可以重定向,
stderr 也是可以重定向的. 为了区别, 我们可以分别用 1(可以忽略) 和 2 表示这两个不同的通道. 比如:
\begin{verbatim}
./count 10 >std.txt 2>err.txt
\end{verbatim}
或者等价地
\begin{verbatim}
./count 10 1>std.txt 2>err.txt
\end{verbatim}
注意这里 \verb|1>| 和 \verb|2>| 之间不要有空格, 否则会和命令行参数混淆.

这几件规则可以结合在一起发挥作用. 比如:
\begin{verbatim}
./count 10 | grep count
\end{verbatim}
这里你仔细观察, 就会发现, 管道只接管了 \verb|std|, 而且管道之后的命令的执行和
\verb|count| 是同步的. 两边的输出甚至可能交错. 这里涉及到进程并行的问题. 我们先不讨论.
所以我们可以先清理一下输出, 比如把 \verb|stderr| 先重定向掉. 
\begin{verbatim}
./count 10 2>/dev/null | grep count
\end{verbatim}

\subsection{shell 编程}

我们可以做的更过份一点. 比如我们知道在 \verb|/usr/include| 下放了很多 C 语言的头文件,
比如 stdio.h 就在这里. 现在我们想看一下这些头文件中有多少包含了 \verb|stdout| 这个关键字.
比如我们想看一下这个文件是在哪个文件定义的. 那么我们可以用命令:

\begin{verbatim}
grep stdout /usr/include/*
\end{verbatim}

找到这些文件, 然后记下来, 再一个个文件去读. 这就有点蠢了. 就不能一次搞定? 找到这些包含
\verb|stdout| 的头文件, 找到一个, 读一个, 这样就很快能发现是谁定义, 还能马上看到定义周围的细节.
要做到这件事, 我们可以写一个段小的脚本:

\begin{verbatim}
$ for file in /usr/include/*
> do
> if grep -l stdio $file
> then
> more $file
> fi
> done
\end{verbatim}

这样就可以愉快地做到了. 注意我们可以用上箭头调用上一次运行的命令, 用 TAB 键自动补齐一个最有可能的命令.
我们这里用一下上箭头, 你会看到原来刚才那一长串都算一个命令.

考虑到这个命令输入太麻烦, 我们不如把它做成一个可执行脚本算了.
查看脚本 \verb|first|. 





\bibliographystyle{plain}
\bibliography{crazyfish.bib}

\end{document}
