%!Tex Program = xelatex
%\documentclass[a4paper]{article}
\documentclass[a4paper]{ctexart}
\usepackage{xltxtra}
\usepackage{url}
%\setmainfont[Mapping=tex-text]{AR PL UMing CN:style=Light}
%\setmainfont[Mapping=tex-text]{AR PL UKai CN:style=Book}
%\setmainfont[Mapping=tex-text]{WenQuanYi Zen Hei:style=Regular}
%\setmainfont[Mapping=tex-text]{WenQuanYi Zen Hei Sharp:style=Regular}
%\setmainfont[Mapping=tex-text]{AR PL KaitiM GB:style=Regular} 
%\setmainfont[Mapping=tex-text]{AR PL SungtiL GB:style=Regular} 
%\setmainfont[Mapping=tex-text]{WenQuanYi Zen Hei Mono:style=Regular} 


\title{第七讲: 图像的输出和进阶科学计算软件}
\author{王何宇}
\date{}
\begin{document}
\maketitle
\pagestyle{empty}

\section{The Tik Z and PGF Packages}

这个包已经包含在 texlive-full 中. 只需直接使用即可. 所谓矢量图形, 表示图形的元素都是公式定义的,
在显示时由显示软件或格式 (pdf, eps) 动态生成, 因此可以无极缩放, 不会出现锯齿.
所以矢量图形具有极高的显示和打印质量, 并且存储量极小. 我们在撰写文章时, 应该尽量采用这种形式.
和矢量图形相对的, 就是点阵图形, 我们已经掌握了它们的基本规则, 就是位图. 位图一旦缩放,
就必然会出现图像失真和锯齿. 

\begin{itemize}
\item 项目主页: \url{https://github.com/pgf-tikz/pgf}
\item 学习资料: 群内文件: \verb|LatexMan/tikzpgfmanual.pdf|
\item 学习方式: 快速阅读并重现: Part I Tutorials and Guidelines 部分的全部例子.
\item 辅助应用: 源内有应用: tikzit 可以帮助学习如何绘制图形.
\item 资源网站: \url{https://texample.net/tikz/examples/}
\end{itemize}

\section{全面的绘图工具: gnuplot}

Gnu 出品必属精品. 能以极高效率产生各种图形, 并且能够对接各种可能的应用. 它可以和 latex 和 tikz
配合使用. 安装下载既可以从项目主页获得最新源码编译, 也可以直接从源里安装:
\begin{verbatim}
sudo apt install gnuplot
\end{verbatim}

\begin{itemize}
\item 项目主页: \url{http://www.gnuplot.info/}
\item 学习资料: \verb|sudo apt get install gnuplot-doc|
\item 辅助资料: 群内文件: \verb|数学软件| 目录下有资料.
\item 学习方式: \verb|/usr/share/doc/gnuplot/tutorial.pdf| 仔细看一遍.
\item 资源网站: \url{http://www.gnuplot.info/demo/}
\end{itemize}

这个软件甚至具有很强的计算功能!

\section{高级科学计算软件: deal.II}

这是一个已经开发了 20 余年的结构化自适应有限元软件包. 计算能力非常强大,
可以从事科研和工业级别计算. 它的一个优点是提供了全面的文档, 例子和学习资料.
这个软件包新手编译安装较为困难, 建议直接安装源内项目:
\begin{verbatim}
sudo apt install libdeal.ii-dev
sudo apt install libdeal.ii-doc
\end{verbatim}
由于大量消耗计算资源, 不建议在虚拟机下使用.
它的最合适硬件环境应该是游戏本级别以上的工作站或中小型集群.

\begin{itemize}
\item 项目主页: \url{https://www.dealii.org/}
\item 学习资料: 安装 doc 后, 在 \verb|/usr/share/doc/libdeal.ii-doc/| 目录下,
  有 \verb|examples| 和 \verb|html| 两个目录, 复制到个人文件夹下.
\item 辅助资料: 视频课程: \url{https://www.bilibili.com/video/av57103047/}
\item 学习方式: 先去学一遍有限元, 然后将 \verb|html| 中的 tutorial 看一遍,
  再将 \verb|examples| 中程序跑一遍. 
\end{itemize}

\bibliographystyle{plain}
\bibliography{crazyfish.bib}

\end{document}
